\section{Research Goals}
\label{sec:introduction:research_goals}

This study aims to develop a processing pipeline that both detects and recognises \glspl{rbn} on a marathon runner, and then ranks the prominence of each runner detected in the photo. The intention is to explore the viability of artificial deep-learning \glspl{nn}---such as \glspl{cnn}---in the pipeline. Previous studies in \gls{rbn} recognition \citep{Benami:2012jf} and similar areas \citep{Kundu:2015vq, Eichner:2008dw, Torresen:2004jl} were heavily heuristic and rule driven.

This primary aim is developed into three key objectives:

%By working with a large labelled dataset in the context of RBN recognition, a benchmark between existing libraries and open source tools can be assessed using techniques developed in other related works. A collated literature review of the state of the art in image processing and image segmentation from other similar areas (such as speed limit sign detection) is to be conducted. The study will also attempt to train a neural network to recognise entire numbers in an RBN, rather than segmented numbers joined together.

%When compared to the wider context of alphanumeric text recognition, this study focuses solely in the area of RBN detection. A generalised approach may be evolved from the study, such as applying the pipeline to other subjects or moving images (such as parked cars).


\subsubsection*{Goal 1: \itshape Detect \glspl{rbn} using a \gls{cnn}}

% 1: Explain the hypothesis
Literature has shown that heuristic-based detection algorithms (that are \gls{cc}-based) are able to detect text within photos \citep{Li:2012wd, Chen:2011ul, Eichner:2008dw}. We propose to apply these rule-based techniques to a large labelled dataset within the context of \gls{rbn}, and contrast them against a learning-based detection and recognition algorithms (using \glspl{nn}). By developing an end-to-end recognition pipeline, we explore if learning-based detection methods can detect these sequences just as heuristic-based detection algorithms (focusing namely on \gls{cc}-based detection) can.
% 2: Suggest the research question from the hypothesis
For this goal the research questions are framed as:
\begin{enumerate}[label=\bfseries~RQ\arabic*), leftmargin=2cm, rightmargin=1.5cm]
  \item\label{rq:1} Can \glspl{cnn} be taught to recognise \glspl{rbn}?
  \item\label{rq:2} What is a systematic methodology for developing annotated data required for training the \gls{cnn}?
\end{enumerate}
% 3: Explain what phenomenon that will be observed
% 4: Explain possible evidence that shows this phenomenon
%The findings from these experiments offer metrics (namely, recall and precision rates) to compare the merit of learning-based versus \gls{cc}-based detection methods within our dataset of marathon photos. We observed that the precision and recall rates of learning-based detection methods compared \todo{favourably$~|~$worse than} those of \gls{cc}-based methods.

\subsubsection*{Goal 2: \itshape Use a \gls{nn}-based \gls{ocr} engine to recognise the detected \gls{rbn}}

% 1: Explain the hypothesis
Typically, traditional alphanumeric sequence parsing can be performed by character segmentation, and then piping those characters into \glsx{ocr} engines.
% 2: Suggest the research question from the hypothesis
In the context of marathon photos, we explore answers to the following:
\begin{enumerate}[label=\bfseries~RQ\arabic*), leftmargin=2cm, rightmargin=1.5cm]
  \setcounter{enumi}{2}
  \item\label{rq:3} Can a \gls{nn}-based \gls{ocr} approach recognise segmented extracted \gls{rbn} detected from an image?
  \item\label{rq:4} Does a \gls{cnn}-based \gls{ocr} algorithm perform \textit{without} the use of character segmentation?
\end{enumerate}
% 3: Explain what phenomenon that will be observed
% 4: Explain possible evidence that shows this phenomenon
% The findings from these experiments observed that the development of our learning-based \gls{ocr} pipeline outperformed that of a traditional \gls{ocr} engine by a factor of \todo{value}.

\subsubsection*{Goal 3: \itshape Defining prominence of alphanumeric sequences}

% 1: Explain the hypothesis
Our research objective focuses on whether prominence of \glspl{rbn} can be captured and then fed into \gls{nn}.
% 2: Suggest the research question from the hypothesis
We can therefore propose the followings research questions:
\begin{enumerate}[label=\bfseries~RQ\arabic*), leftmargin=2cm, rightmargin=1.5cm]
  \setcounter{enumi}{3}
  \item\label{rq:5} How can quantifiable prominence be collected for the purposes of data collection?
\end{enumerate}
% 3: Explain what phenomenon that will be observed
% 4: Explain possible evidence that shows this phenomenon
%The findings from these experiments observed that the development of a ranking system for \glspl{rbn} was able to match human characteristics with a similarity factor of \todo{value}.
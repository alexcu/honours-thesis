\section{Motivation}
\label{sec:motivation}

Detection becomes difficult when the photo is unstructured. Early investigations in LPR systems were systematic in the subject material assessed; a detailed survey by \cite{Anagnostopoulos:2008vu} showed that they work best  with consistent lighting, specific colour and typeface detection, fixed detection regions, and non-noisy backgrounds. When applied in the context of images with unstructured backgrounds, these systematic approaches begin to have sever limitations as the text components cannot be easily determined.

While further investigations in the area utilise enhanced CC-based detection \citep{Chen:2011ul,Shivakumara:2011dl,Epshtein:2010tj}, performance is likely to degrade as image complexity increases \citep{Li:2012wd}. This is especially relevant when text is geometrically obfuscated, such as malformed RBNs as worn on a marathon runner's torso. Some studies have shown to overcome this by using facial recognition to find a more distinct candidate area \citep{Benami:2012jf}, but nonetheless relies on such properties like a person's face to detect a number. Similarly, most recognition techniques interpret text as segmented characters, rather than a single string, though there are exceptions such as in \cite{Zhu:2016ut}.

We therefore identify gaps using a learning-based approach in \emph{both} detection and recognition using deep-learning artificial NNs. We also identify gaps in prominence ranking that may also potentially use NNs to order prominence of runners based from their detected RBN.

%By working with a large labelled dataset in the context of RBN recognition, a benchmark between existing libraries and open source tools can be assessed using techniques developed in other related works. A collated literature review of the state of the art in image processing and image segmentation from other similar areas (such as speed limit sign detection) is to be conducted. The study will also attempt to train a neural network to recognise entire numbers in an RBN, rather than segmented numbers joined together.

%When compared to the wider context of alphanumeric text recognition, this study focuses solely in the area of RBN detection. A generalised approach may be evolved from the study, such as applying the pipeline to other subjects or moving images (such as parked cars).
\section{Motivation}
\label{sec:motivation}

Detection becomes difficult when the photo is unstructured. Early investigations in LPR systems were systematic in the subject material assessed; a detailed survey by \cite{Anagnostopoulos:2008vu} showed that they work best  with consistent lighting, specific colour and typeface detection, fixed detection regions, and non-noisy backgrounds. When applied in the context of images with unstructured backgrounds, these systematic approaches begin to have sever limitations as the text components cannot be easily determined.

While further investigations in the area utilise enhanced CC-based detection \citep{Chen:2011ul,Shivakumara:2011dl,Epshtein:2010tj}, performance is likely to degrade as image complexity increases \citep{Li:2012wd}. This is especially relevant when text is geometrically obfuscated, such as malformed RBNs as worn on a marathon runner's torso. Some studies have shown to overcome this by using facial recognition to find a more distinct candidate area \citep{Benami:2012jf}, but nonetheless relies on such properties like a person's face to detect a number. Similarly, most recognition techniques interpret text as segmented characters, rather than a single string, though there are exceptions such as in \cite{Zhu:2016ut}.
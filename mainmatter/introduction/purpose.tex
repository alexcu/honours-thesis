\section{Motivation}
\label{sec:introduction:motivation}

Detection is harder when the photo is unstructured. Early investigations in \glsx{lpr} systems were systematic in the subject material assessed; a detailed survey by \cite{Anagnostopoulos:2008vu} showed that they work best  with consistent lighting, specific colour and typeface detection, fixed detection regions, and non-noisy backgrounds. When applied in the context of images with unstructured backgrounds, these systematic approaches begin to have limitations as the text components cannot be easily determined.

While further investigations in the area utilise enhanced \glsx{cc}-based detection \citep{Chen:2011ul,Shivakumara:2011dn,Epshtein:2010tj}, performance is likely to degrade as image complexity increases \citep{Li:2012wd}. This is especially relevant when text is geometrically obfuscated, such as malformed \glsplx{rbn} as worn on a marathon runner's torso. Malformed, in this sense, is caused by non-flat bib sheets that tend to follow the runner's body shape, in addition to images that are taken in dynamical contexts. Some studies have shown to overcome this by using facial recognition to find a more distinct candidate area \citep{Benami:2012jf}, but nonetheless rely on a person's face to detect a number. Similarly, typical recognition techniques interpret text as segmented characters, rather than a single string, though there are exceptions such as in \citet{Zhu:2016ut}.

We also identify subject prominence ranking within natural scenes as an area that has little exploration within literature. (For example, the prominence of a \textit{specific} marathon runner within a scene of many runners.) Prominence ranking is an important field in the context of \gls{rbn} recognition: runners typically choose not to purchase photos where they have been recognised in an image but are not in the foreground. There are also varying factors which influence purchase likelihood, such as face visibility, eye contact with the camera, and blurriness. An assessment into how the prominence of a runner can be ordered in hundreds of identified photos (based from their recognised \gls{rbn}) can be used by use of a \glsx{nn}.

This study forms part of an industry project under the \gls{dstil}. As a part of the research project, access has been made to a labelled dataset of hundreds of thousands of marathon photos.

%When compared to the wider context of alphanumeric text recognition, this study focuses solely in the area of RBN detection. A generalised approach may be evolved from the study, such as applying the pipeline to other subjects or moving images (such as parked cars).
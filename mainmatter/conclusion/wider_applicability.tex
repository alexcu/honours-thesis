\section{Wider Applicability}

Alphanumeric character recognition goes far beyond \gls{rbn} recognition. Other contexts in the area of self-driving vehicle literature, such as \gls{lpr} and \glsx{tsr}, is imperative for the realisation of these vehicles. Similar areas include meter reading of electronic and water meters, or street sign numbers as discussed in \cref{ch:background}. 

The objective is to change the feature of interest: if one is interested in training a network about license plates, then train the network as we have done with racing bib numbers, but with an annotated dataset labelled with license plates. Additionally, training a text detection network with typeface specificity in mind (i.e., if one knows the typeface of a license plate or electronic meter), swapping out the text recognition portion of our pipeline with this trained network is achievable. This can be done to not only improve our \gls{ocr} bottleneck, but to improve context-specific cases.

% Runtime speed improvements
%% Where was the biggest bottleneck? Engineering of Keras FRCNN to be improved architecturally to always be running on a all image basis, not spin up one per image

% OCR improvements
%% Use a NN to detect text better on a per-character basis.
%% Train a NN with domain-specific font. E.g., Vic License Plate typeface, typeface for electric box readers

% Wider applicability (other contexts)
%% Speed limit signs
%% License plate recognition
%% Electric box/water meters
%% Use a NN to train the OCR for font-specific applicability
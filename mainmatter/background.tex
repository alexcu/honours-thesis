\chapter{Background}
\label{ch:background}


Text capturing within unstructured photos from the wild are typically achieved in two stages: detection and recognition. Detection techniques are classified as either \gls{cc}- or learning-based. The recognition phase uses traditional \gls{ocr} engines or, more recently, artificial \glspl{nn}. 

In this chapter, we survey different applications where \gls{rbn} recognition (and related works) are investigated. Various detection and recognition techniques discussed in literature are detailed.

\section{Detection Strategies}
\label{sec:background:detection_strategies}

Text extraction strategies have seen continuing interest in the literature, with many comprehensive surveys assessing the state of the art \cite{Chen:2000ua, Jung:2004uwa, Zhang:2008vfa, Liang:2005uy, Jung:2004uw}. It is widely demonstrated that if text within an unstructured scene is \textit{detected} reliably, then existing \gls{ocr} engines can suitably extract these characters \citep{Smith:2007dc} once they exist in a structured context; thus not every . A survey into the two prominent detection strategies is given in Sections~\ref{sec:detection:cc_based} to \ref{sec:detection:learning_based}.

These two prominent strategies have a varied nomenclature: (1) the \gls{cc}-based (or \textit{region}-based) approach, that utilise different region properties (e.g., colour, edges, \glspl{cc}) \citep{Jain:1998wd, Kim:1996tw, Liu:2006wh, Chen:2011ul, Li:2012wd, Zhang:2011cl, Shivakumara:2011dn, Epshtein:2010tj, Zhang:2010wa, Shivakumara:2010wu, Liu:2008tz, Subramanian:2007tf, Lee:2010vv, Sun:2010tg} for unsupervised extraction; and, (2) learning-based (or \textit{texture}-based) approach, which uses unique texture properties to supervise extraction text from its background \citep{Wang:2009vp, Hanif:2009tm, Tu:2003tg, Ye:2005wu, Lee:2003cn, XiangrongChen:2004ha, Chen:2005wv}. Additionally, some authors have proposed methods to combine these unsupervised and supervised techniques \citep{Mutch:2006ub, Mairal:2008uw, Bengio:2006vb}.

\subsection{CC-based techniques}
\label{sec:detection:cc_based}

\gls{cc}-based approaches generate separated \glspl{cc} using properties such as stroke width, pixel colour and edges, typically applying geometric and texture filters to reduce false positives. Neighbouring pixels are then `grouped' using an algorithm originally presented by \citet{Horn:1986vc}. 

Previous work required the use of a scanning window \citep{XiangrongChen:2004ha, Lienhart:2002ub, Jung:2009do} which is limited by a constant image scale and discrete orientations of the sliding (thereby preventing text strokes in non-linear directions). However, a study by \citet{Epshtein:2010tj} (and coincidentally \citet{Zhang:2011cl}) introduced the concept of \gls{swt}, a local image operator that determines the most likely stroke of a given pixel by computing the per-pixel width. In this study, edges are initially computed using the Canny-Edge Detection algorithm \citet{Canny:1986uw}

\subsection{Learning-based techniques}
\label{sec:detection:learning_based}
typically referred to as a \textit{learning}-based approach, due to the common use of machine learning methods utilised

Typically, texture-based approaches utilise supervised learning methods, though it is typical for these classifiers to required thousands of training images \cite{Chen:2004ux}. Additionally, these methods
\section{Recognition Strategies}
\label{sec:background:recognition_strategies}

The \gls{icdar} robust reading competitions \citep{Lucas:2003iw, Lucas:2005bq, Shahab:2011hq} broke down the issue of text extraction into two sub-problems: text locating and character recognition. Most of the literature discussed in Section~\ref{sec:background:detection_strategies} focused within the text locating sub-problems; there were no expressions of interest in character recognition in \gls{icdar} 2003 and 2005 and only three in 2011 (the top system \citep{Liu:2005uw} scoring a correct recognition rate of 41.2\%). \todo{Refer to ICDAR 2013 and ICDAR 2015}.

It has been widely demonstrated that off-the-shelf commercial and open source OCR packages are able to correctly recognise text once the characters are extracted. Such usages include the use of the Tesseract \gls{ocr} \citep{Benami:2012jf} \todo{cite ICDAR2015}, TOCR and Readiris Pro \gls{ocr} \citep{XiangrongChen:2004ha}, ABBYY Fine Reader \gls{ocr} \citep{XiangrongChen:2004ha, Gatos:2005wd, Wang:2011tw, XiaomingHuang:2015fb}, TH-OCR \todo{cite from ICDAR2011}, INZI \todo{cite from ICDAR2011}, and TypeReader \gls{ocr} \citep{Li:2010dy} engines \todo{Link to relevant URLs for OCR engines}. Recent interest in developing novel general character recognition strategies has developed, such as the use of lexicon-free photo \gls{ocr} frameworks \citep{Lee:2016uy}.

\begin{figure}[h]
  \centering
  \includegraphics[width=0.75\textwidth]{images/background/kundu2015_nn}
  \caption[A NN designed to recognised speed limit signs]{The artificial \glsx{mlp} \gls{nn} designed in \citet{Kundu:2015vq} to recognise US-style speed limit signs.}
  \label{fig:background:recognition:kundu2015_nn}
\end{figure}

A \citeyear{Kundu:2015vq} study into \glsplx{tsr} to detect US-style speed limit signs achieved recognition without the use of any OCR packages. In \citep{Kundu:2015vq}, \citeauthor{Kundu:2015vq} were able to extract a speed limit sign via the use of \glspl{mser} and template matching. The resulting detected signs were scaled to a grayscaled size of $20 \times 25$ pixels and fed into a \glsx{mlp} \gls{nn} of 200 neurons in the hidden layer. The output layer of the network consisted of seven nodes, each representing the seven kinds of speed limit signs in US cities (25, 30, 35, 40, 45, 50, and 55 miles per hour). This architecture is shown in Figure~\ref{fig:background:recognition:kundu2015_nn}. When trained with 13,289 images of text cases and 4,319 non-text cases, the results showed that their recognition classier was able to correctly recognise speed limit signs with an accuracy of 98.04\%. Similar results were achieved using a feed-forward \gls{mlp} in \cite{Eichner:2008dw}, using UK/Poland style speed limits scaled to $20 \times 20$ pixels (grayscale) and 12 output layer neurons ($10 \dots 100$ kilometres per hour, the national speed limit sign, and non-sign neurons).

However, works in \gls{tsr} systems that utilise networks are generally non-generalisable, and only work in a limited context (i.e., by classifying speed limit signs of known outputs). In our context of \gls{rbn} recognition, we have a known character output range of 36 possibilities (0--9 and A--Z = 10 + 26). Beyond \gls{tsr} systems, however, we see the use of more generalisable networks: \citet{Netzer:2011to} trained neural network to recognise street number characters from Google Street View with higher precision and recall than that of \gls{hog} and the Tesseract \gls{ocr} engine, showing that the applicability of \glspl{nn} for recognition can outperform traditional means. \citet{Anagnostopoulos:2006wv} used a \gls{pnn} to recognise single characters of the same 36 possibility range (i.e., uppercase alphanumeric characters) corresponding to the input grayscale vector of $9 \times 12$ pixels ($9 \times 12 = 108$ input neurons) for a single character. Figure~\ref{fig:background:recognition:anagnostopoulos2006_nn} illustrates the \gls{pnn} architecture used in this study. Furthermore, investigations in comparing different architectures of \glspl{nn} for this context is given in \citet{Lee:2016uy}.

\begin{figure}[h]
  \centering
  \includegraphics[width=0.75\textwidth]{images/background/anagnostopoulos2006_nn}
  \caption[A PNN used to recognise license plate characters]{\citet{Anagnostopoulos:2006wv} developed the architecture of a \gls{pnn} to determine a single character from within a license plate.}
  \label{fig:background:recognition:anagnostopoulos2006_nn}
\end{figure}

% Talk about BIB literature
% Talk about number plate literature
% Talk about speed sign NN literature
\section{Metrics}
\label{sec:background:metrics}

Throughout our survey, we have utilised the evaluation scheme first proposed for use in image processing in the ICDAR text extraction competitions \citep{Lucas:2003iw, Lucas:2005bq, Shahab:2011hq}. This scheme was designed to be easy to understand and compute, reward text extraction useful for natural scenes, and heavily punish trivial solutions. The intention behind these metrics were to develop a measure of `robustness' a text extraction pipeline can achieve.

\subsection{Precision and Recall}
\label{sec:background:metrics:precision_and_recall}

Generally in information retrieval, the precision ($p$) and recall ($r$) metrics are used, first defined in the six evaluation criteria for information retrieval systems by \citet{Cleverdon:1966vd}. Precision refers to the proportion of relevant matches actually retrieved in the retrieved results, while recall refers to the proportion of relevant matches retrieved in total relevant instances. We use recall and precision metrics to assess the \textit{effectiveness} of an information retrieval system \citep{Rijsbergen:1979dw}.

In the context of image processing, systems that over-estimate are punished with a low precision score, while systems that under-estimate are punished with a low recall score \citep{Lucas:2003iw}. Therefore, precision is the number of correct candidates ($c$) divided by the number of total estimates found ($E$):
\begin{equation*}
  p = \frac{c}{\lvert\;E\;\rvert}
\end{equation*}

And recall is defined as the number of correct estimates divided by the total number of ground-set truth targets ($T$):
\begin{equation*}
  r = \frac{c}{\lvert\;T\;\rvert}
\end{equation*}

However, it is not realistic for a given text extraction pipeline to \textit{exactly} agree with the rectangle bounds manually tagged by a human. \citet{Lucas:2003iw} first proposed changes to these calculations to better suit their usage in the context of information extraction from within images. They adopt a more flexible notion of what a `match' is. They define a new match measure ($m_{p}$) between two rectangles (i.e., the ground truth and the system's detected candidate) as ``the area of intersection of both rectangles divided by the area of the minimum bounding box containing both rectangles'' \citep{Lucas:2003iw}. This allows for a match value of one when the candidate is identical to the ground truth, and zero where the candidate has no intersection at all to the ground truth.

Therefore, the best match, $m(r,\,R)$, of a rectangle $r$ in a set of rectangles $R$ is:
\begin{equation*}
  m(r,\,R) = \mathrm{max}~m_{p}(r,\,r')~|~r' \in R
\end{equation*}

Lastly, we can redefine the recall and precision metrics to be more forgiving in the image extraction context:
\begin{align*}
  p' &= \frac{\sum\,_{r_{e}\;\in\;E}~m(r_{e},\,T)}{\lvert\;E\;\rvert}\\ \\
  r' &= \frac{\sum\,_{r_{t}\;\in\;T}~m(r_{t},\,E)}{\lvert\;T\;\rvert}
\end{align*}

\subsection{The \fscore}
\label{sec:background:metrics:fscore}

Common metrics used when developing text extraction pipelines utilise the use of the \fscore, a single measure of quality that combines both precision and recall values computed above. We are able to compute this metric using the standard measure across many studies, as contrasted in \todo{reference the table}.

The \fscore{} algorithm is given in the context of image processing in \citet{Lucas:2003iw}. Relative weights controlled by an $\alpha$ value of 0.5 give equal weight to both precision and recall metrics:
\begin{equation*}
  f = \frac{1}{\frac{\alpha}{p'} + \frac{1-\alpha}{r'}}
\end{equation*}

% QUOTE: Since it is unlikely to produce estimated rectangles which exactly align with the manually labeled ground truth, the f metric can vary from 0.8 − 1.0 even when all text is correctly localized.

\newpage
\section{Conclusion}

In this chapter, we have presented a survey of literature in various application contexts: \gls{rbn} and \gls{tsr} recognition, recognition of alphanumeric sequences `in the wild', and additionally object instance segmentation. We also present the varied range of techniques used to both detect and recognise the text, using both heuristic-based and \gls{nn}-based approaches. This said, we acknowledge that datasets within the survey are not always recent (ranging as far back as \citeyear{Lucas:2003iw}) due to tendencies for researchers to use the more popular (though aged) \gls{icdar} datasets.

The state of the art of learning-based detection approaches such as \glspl{cnn} for have gained wide popularity, albeit for object segmentation. These approaches are yet to be applied within the context of alphanumeric sequences. Recent years have had a heavier focus on heuristic-based detection strategies using \gls{cc}-based methods, while a majority of learning-based detection methods have had far fewer recent investigations. Furthermore, recognition of characters using \glspl{nn} are not yet widely used, and off-the-shelf \gls{ocr} packages are still standard.
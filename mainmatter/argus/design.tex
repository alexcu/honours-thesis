\section{Design}

Argus is designed as a full-screen desktop application to utilise maximum screen space. The application is designed to be keyboard-driven, so keyboard shortcuts are displayed wherever possible. Additionally, the current instruction of the workflow is indicated on the top of the image in red to emphasise to annotators what input they are required to make. \cref{fig:dataset:argus:overview} shows the main \gls{ui}. Further segment-level features are captured in dialogs or user interaction directly on the image, as shown in \cref{fig:dataset:argus:bib_and_face,fig:dataset:argus:prom_and_col}. We deployed Argus to data taggers remotely using the ClickOnce Deployment\footnoteurl{https://msdn.microsoft.com/en-us/library/t71a733d.aspx}{11 August 2017} strategy.

\begin{figure}[h]
  \centering
  \includegraphics[width=\textwidth]{images/dataset/argus/argus_ui}
  \caption[An overview of the Argus user interface]{An overview of the Argus user interface.}
  \label{fig:dataset:argus:overview}
\end{figure}

\begin{figure}
  \centering
  \includegraphics[width=\textwidth]{images/dataset/argus/argus_entry}
  \caption[Bib and Face feature annotation with Argus]{$Bib$ and $Face$ segment-level feature annotation. Users click four times around the bib to mark up the $BibSheet$ region (left). A dialog asks the user to enter the $\gls{rbn}$ label (top). The \gls{rbn} is annotated on the image (middle) and users can progress to drag-and-drop around the $Face$ region within the restrictions set (see \cref{sec:dataset:architecture:metamodel}). Note the dependency ordering is present.}
  \label{fig:dataset:argus:bib_and_face}
\end{figure}

\begin{figure}
  \hspace{\fill}
  \begin{subfigure}[b]{0.45\textwidth}
    \includegraphics[width=\textwidth]{images/dataset/argus/argus_prom_entry}
  \end{subfigure}
  \hspace{\fill}
  \begin{subfigure}[b]{0.45\textwidth}
    \includegraphics[width=\textwidth]{images/dataset/argus/argus_color_entry}
  \end{subfigure}
  \hspace{\fill}
  \caption[Prominence and Colour feature annotation with Argus]{Annotation for the $Prominence$ and $Colour$ segment-level features.}
  \label{fig:dataset:argus:prom_and_col}
\end{figure}
\chapter{Introduction}
\label{ch:introduction}

Ever since the camera and phone were unified into smartphones, we have seen an increasing interest for image understanding (specifically to identify the content of an image) but text recognition still faces challenges within images of unstructured scenes. While successes in character recognition have a long history with Optical Character Recognition (OCR)\newacronym{ocr}{OCR}{Optical Character Recognition} engines \citep{Smith:1987tg}, these are typically applied under strict conditions (e.g., flatbed scanners for documents without distracting backgrounds). Once applied within the context of a natural scene, real-world discrepancies pose serious shortcomings, such as illumination and viewpoint conditions, blur and glare variations, geometric and photometric distortion, and differences in font size and style. Overcoming these issues has motivated a variety of different techniques in order to realise potential applications that make use of text recognition at scale. 

With the ubiquity of smartphone cameras, practical applications of natural image processing have increased. In the last two decades, we have seen the development of point-and-shoot product recognition \citep{Tsai:2010cn,Girod:2011gw}, object detection in videos \citep{Sivic:2003tj}, building recognition \citep{Takacs:2008cg}, image feature extraction to improve visual-based search engines \citep{Lowe:2004kp,Bay:2008ud}, and translation services of American Sign Language gestures \citep{Jin:2016jd}. Nonetheless, embedded text within images reveals the largest form of informative features about the image; if text extraction is therefore not robust, information extraction suffers.

Text detection robustness is a factor which severely limits a text recognition pipeline. Research in overcoming such limitations were competed in \citet{Lucas:2003iw}, where robustness was a key focus in the image processing pipelines proposed. This focus was reiterated by \citet{Chen:2011ul}, who state the primary prerequisite for text-based recognition (especially within natural scenes) is the text location must be robustly located.

As with any data processing pipeline, false negatives increase where early stages of the pipeline fail, and therefore detection of these potential candidates must be robust. We can reduce errors in a pipeline where: (1) there may be unwarranted stages of the pipeline, and therefore \textit{excluding} unnecessary stages may also assist in reducing error cases, and (2) by piping through unmatched candidates to further pipelines, which can increase the detection. Both guide in improving robust detection, and this is therefore a key consideration made in our assessment of how useful that pipeline may be. Without the construct of robustness, we restrict these pipelines to very confined conditions, and its usefulness in products is not warranted.

\section{Background}

This study presents character recognition within the context of short, alphanumeric number sequences. We define alphanumeric number sequences as short fragments of digits within an unstructured scene.   

Certain literature have focused on these kinds of sequences, namely: License Plate Recognition (LPR\newacronym{lpr}{LPR}{Licence Plate Recognition}) systems \citep{CanoPerez:2003fq, Anagnostopoulos:2008vu}; Traffic Sign Recognition (TSR\newacronym{tsr}{TSR}{Traffic Sign Recognition}), namely speed limit recognition, to better realise Advanced Driver Assistance Systems (ADAS\newacronym{adas}{ADAS}{Advanced Driver Assistance Systems}) \citep{Eichner:2008dw,Kundu:2015vq,Seo:2015ez,Lian:2016dc}; and, street number recognition, specifically two studies by \citet{Netzer:2011to}, using Google Street View\footnoteurl{https://www.google.com/streetview/}{13 May 2017} to determine the numerical value of street numbers. \todo{Figure~\ref{}} summarises the usage of these sequences.

Different applications apply varying methods to parse short alphanumeric characters. There are typically two stages of any parsing method: detection of possible candidates and recognition of the text itself. Detection techniques usually are categorised as either connected component (CC)\newacronym{cc}{CC}{Connected Component}-based or learning or texture-based. CC-based detection will typically use a set of distinct properties on the image to detect relevant areas (such as width, stroke and colour) while learning-based feed images into a classifier that can distinguish candidates from false positives. The recognition phase can typically be achieved using optical image recognition OCR engines, machine learning algorithms or deep neural networks to classify the detected regions.

{
  \itshape
  This study proposes the development of a learning-based detection and recognition pipeline using deep-learning neural networks within the context of unstructured photos, namely focusing within the context of marathon Racing Bib Numbers (RBNs)\newacronym{rbn}{RBN}{Racing Bib Number} (\todo{Figure~\ref{}}).
}

\section{Motivation}
\label{sec:motivation}

Detection becomes difficult when the photo is unstructured. Early investigations in LPR systems were systematic in the subject material assessed; a detailed survey by \cite{Anagnostopoulos:2008vu} showed that they work best  with consistent lighting, specific colour and typeface detection, fixed detection regions, and non-noisy backgrounds. When applied in the context of images with unstructured backgrounds, these systematic approaches begin to have sever limitations as the text components cannot be easily determined.

While further investigations in the area utilise enhanced CC-based detection \citep{Chen:2011ul,Shivakumara:2011dl,Epshtein:2010tj}, performance is likely to degrade as image complexity increases \citep{Li:2012wd}. This is especially relevant when text is geometrically obfuscated, such as malformed RBNs as worn on a marathon runner's torso. Some studies have shown to overcome this by using facial recognition to find a more distinct candidate area \citep{Benami:2012jf}, but nonetheless relies on such properties like a person's face to detect a number. Similarly, most recognition techniques interpret text as segmented characters, rather than a single string, though there are exceptions such as in \cite{Zhu:2016ut}.

We therefore identify gaps using a learning-based approach in \emph{both} detection and recognition using deep-learning artificial NNs. We also identify gaps in prominence ranking that may also potentially use NNs to order prominence of runners based from their detected RBN.

%By working with a large labelled dataset in the context of RBN recognition, a benchmark between existing libraries and open source tools can be assessed using techniques developed in other related works. A collated literature review of the state of the art in image processing and image segmentation from other similar areas (such as speed limit sign detection) is to be conducted. The study will also attempt to train a neural network to recognise entire numbers in an RBN, rather than segmented numbers joined together.

%When compared to the wider context of alphanumeric text recognition, this study focuses solely in the area of RBN detection. A generalised approach may be evolved from the study, such as applying the pipeline to other subjects or moving images (such as parked cars).
\section{Research Goals}
\label{sec:research_goals}

This study aims to develop a processing pipeline that both detects and recognises RBNs on a marathon runner, and then ranks the prominence of each runner detected in the photo. Using artificial deep-learning neural networks (e.g., convolutional neural networks or CNNs\newacronym{cnn}{CNN}{Convolutional Neural Network}) in this pipeline is the main objective, unlike previous studies in RBN recognition that were heavily heuristic and rule driven. This primary aim is developed into three key objectives:

%By working with a large labelled dataset in the context of RBN recognition, a benchmark between existing libraries and open source tools can be assessed using techniques developed in other related works. A collated literature review of the state of the art in image processing and image segmentation from other similar areas (such as speed limit sign detection) is to be conducted. The study will also attempt to train a neural network to recognise entire numbers in an RBN, rather than segmented numbers joined together.

%When compared to the wider context of alphanumeric text recognition, this study focuses solely in the area of RBN detection. A generalised approach may be evolved from the study, such as applying the pipeline to other subjects or moving images (such as parked cars).

% TODO: A general overview of my research goals

\subsubsection*{Goal 1: \itshape Detect RBNs using a CNN}

% Can a CNN detect an RBN consistently given our dataset?

% 1: Explain the hypothesis

% 2: Suggest the research question from the hypothesis
% 3: Explain what phenomenon that will be observed
% 4: Explain possible evidence that shows this phenomenon

\blindtext

% TODO: Talk about goal 1 in detail

\subsubsection*{Goal 2: \itshape Design a CNN that recognises RBN sequences without character segmentation}

% Can a CNN recognise an entire number sequence without character segmentation?

% 1: Explain the hypothesis
% 2: Suggest the research question from the hypothesis
% 3: Explain what phenomenon that will be observed
% 4: Explain possible evidence that shows this phenomenon

\blindtext

% TODO: Talk about goal 2 in detail

\subsubsection*{Goal 3: \itshape Rank prominence of alphanumeric sequences}

% 1: Explain the hypothesis
% 2: Suggest the research question from the hypothesis
% 3: Explain what phenomenon that will be observed
% 4: Explain possible evidence that shows this phenomenon

\blindtext

% TODO: Talk about goal 3 in detail

\section{Thesis Organisation}
\label{sec:introduction:organisation}

This thesis is organised into the chapters as outlined below. An appendix follows with additional supplementary material.

\paragraph{Chapter \ref{ch:background} - Background} Provides an overview of prior studies broadly around the areas of number detection and recognition in image processing and artificial \glspl{nn}.

\paragraph{Chapter \ref{ch:data_set} - Data Set} Describes the data set to be used, data treatment steps, possible techniques in closer depth to develop a number recognition pipeline, and explores ways to develop prominence ranking techniques.

\paragraph{Chapter \ref{ch:benchmarking} - Benchmarking} Collates results of a series of experiments using our dataset amongst existing open source tools and pipelines presented in previous work

\paragraph{Chapter \ref{ch:processing_pipeline} - Processing Pipeline} Discusses the proposed processing pipeline developed that satisfies the aims of this study.

\paragraph{Chapter \ref{ch:findings} - Findings} Outlines the method used for validation and presentation of our results.

\paragraph{Chapter \ref{ch:discussion} - Discussion} Presents implications that were found from the results of our findings and limitations.

\paragraph{Chapter \ref{ch:conclusion} - Conclusions and Future Work} Draws a number of conclusions and alleviates gaps in the findings of this work by presenting future studies.

\section*{Summary}

In this chapter we identified some shortcomings in text recognition, developed the context of the study---namely RBN detection. We discussed the general stages that exist for text parsing within natural scenes, detection and recognition, and introduced typical techniques that are applied in this context. We outlined the research aims this study achieves, and how the thesis is organised. The following chapter will detail applications of image processing, using neural networks for image processing, and outline what techniques have been used in previous studies to achieve this.
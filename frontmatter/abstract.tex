\chapter*{Abstract}
\addcontentsline{toc}{chapter}{Abstract}

% General statement introducing the broad research area
Text detection in natural images is a growing area with increasing applications, including traffic sign and license plate recognition, and text-based image search.
% An explanation of the specific problem (difficulty, obstacle, challenge)
Robustly detecting and recognising text is especially challenging when text is deformed, such as the photometric and geometric distortions of text worn by a moving subject in unstructured scenes.
% Review of existing or standard solutions to this problem
Existing methods of text detection in such cases are classified as learning-based or connected component (CC)-based, applying a mix of enhanced detection techniques---such as stroke width transformation (SWT), canny-edge detection and maximally stable extremal regions (MSERs)---and feeding candidates into optical character recognition (OCR) engines or neural networks to recognise the text.
% Outline of the proposed new solution
This study proposes applying a learning-based approach using deep-learning strategies and transfer learning to automate the recognition of racing bib numbers (RBNs) in a natural image dataset of various marathons, with the intention to then rank the subject's photos in order of prominence.
% Summary of how the solution was evaluated and what the outcomes of the evaluation were
Experimental results showed that these deep-learning strategies performed favourably, with RBN detection accuracy beyond 95\%. This prompts further investigation in the generality of the technique developed to other similar subject material.
% Quantify the results!!!